\documentclass{article}
\usepackage[utf8]{inputenc}
\usepackage[top=25mm, bottom=20mm, right=25mm, left=25mm]{geometry}
\usepackage{amsmath}
\usepackage{graphicx}
\usepackage{hyperref}

\title{A Simple LaTeX Document}
\author{Your Name}
\date{\today}

\begin{document}

\maketitle

\begin{abstract}
TEST TEST TNST. This is a simple example of a LaTeX document. THIS IS AN EDIT. It includes basic elements such as sections, lists, equations, and images.
\end{abstract}

\section{Introduction}

This is the introduction section $x+y$. LaTeX is a powerful tool for typesetting documents. It is widely used for scientific and technical documents.

\section{Basic Elements}

\subsection{Lists}

You can create lists in LaTeX. Here is an example of an itemized list:
\begin{itemize}
    \item First item
    \item Second item
    \item Third item
\end{itemize}

And here is an example of a numbered list:
\begin{enumerate}
    \item First item
    \item Second item
    \item Third item
\end{enumerate}

\subsection{Mathematics}

LaTeX is particularly good at typesetting mathematics. Here is an example of an inline equation: \( E = mc^2 \).

And here is an example of a displayed equation:
\[
\int_{a}^{b} x^2 \, dx
\]

\subsection{Figures}

You can also include images in your document. Here is an example:
\begin{figure}[h]
    \centering
    \caption{An example image.}
    \label{fig:example}
\end{figure}

\section{Conclusion}

This is the conclusion section. LaTeX provides a lot of flexibility and power for typesetting documents. You can find more information and examples in the \href{https://www.latex-project.org}{LaTeX documentation}.

\end{document}
